\begin{frame}
\frametitle{What is Functional Programming?}
\begin{block}{Let's start at a concrete example}
How do I sum the integer values in a list?
\end{block}
\end{frame}

\begin{frame}[fragile]
\frametitle{What is Functional Programming?}
\begin{block}{Using a for loop}
\begin{lstlisting}[style=java]
sum(list) {
  var r = 0;
  for(int i = 0; i < list.length; i++) {
    r = r + list[i];
  }
  return r;
}
\end{lstlisting}
\end{block}
\end{frame}

\begin{frame}[fragile]
\frametitle{What is Functional Programming?}
\begin{block}{Using a for loop}
\begin{lstlisting}[style=java]
sum(list) {
  var r = 0;
  for(int i = 0; i < list.length; `i++`) {
    `r = r + list[i]`;
  }
  return r;
}
\end{lstlisting}
\end{block}
\end{frame}

\begin{frame}
\frametitle{What is Functional Programming?}
\begin{center}
Here is another way of looking at the problem
\end{center}
\end{frame}

\begin{frame}
\frametitle{What is Functional Programming?}
\begin{block}{The sum of a list is \ldots}
\begin{itemize}
\item if the list is empty, return \lstinline{0}
\item otherwise add the first element to the sum of the remainder of the list
\end{itemize}
\end{block}
\end{frame}

\begin{frame}[fragile]
\frametitle{What is Functional Programming?}
\begin{block}{The sum of a list is \ldots}
\begin{lstlisting}
sum([6, 5, 9, 71, 3]) =
6 + sum ([5, 9, 71, 3]) =
6 + 5 + sum([9, 71, 3]) =
6 + 5 + 9 + sum([71, 3]) =
6 + 5 + 9 + 71 + sum([3]) =
6 + 5 + 9 + 71 + 3 + sum([]) =
6 + 5 + 9 + 71 + 3 + 0 =
94
\end{lstlisting}
\end{block}
\end{frame}

\begin{frame}[fragile]
\frametitle{What is Functional Programming?}
\begin{block}{Here is the Haskell source code}
\begin{lstlisting}[style=haskell]
sum [] = 0
sum (first:rest) = first + sum rest
\end{lstlisting}
\end{block}
\end{frame}
