\begin{frame}[fragile]
\frametitle{Fast and loose reasoning}
\begin{block}{Some programming languages have escape hatches \ldots}
\begin{itemize}
\item \lstinline{null}
\item \lstinline{exceptions}
\item \lstinline{type-casting}
\item \lstinline{type-casing} e.g. \lstinline{instanceof}
\item \lstinline{non-termination}
\end{itemize}
\end{block}
\end{frame}

\begin{frame}[fragile]
\frametitle{Fast and loose reasoning}
\begin{block}{We can (reasonably) disregard these}
\begin{center}
\begin{quotation}
Functional programmers often reason about programs as if
they were written in a total language, expecting the results
to carry over to non-total (partial) languages. We justify
such reasoning.
\end{quotation}
\end{center}
\end{block}
\tiny{Danielsson, Hughes, Jansson \& Gibbons \cite{danielsson2006fast}}
\end{frame}
